\pagestyle{fancy}
\renewcommand\headrulewidth{0pt}
\fancyhf{}
\fancyfoot[R]{\thepage}

\fancypagestyle{plain}{
    \renewcommand{\headrulewidth}{0pt}
    \fancyhf{}
    \fancyfoot[R]{\thepage}
}

\definecolor{pblue}{rgb}{0.13,0.13,1}
\definecolor{pgreen}{rgb}{0,0.5,0}
\definecolor{pred}{rgb}{0.9,0,0}
\definecolor{pgrey}{rgb}{0.46,0.45,0.48}

\lstset{language=Java,
  showspaces=false,
  showtabs=false,
  breaklines=true,
  showstringspaces=false,
  breakatwhitespace=true,
  commentstyle=\color{pgreen},
  keywordstyle=\color{pblue},
  stringstyle=\color{pred},
  basicstyle=\ttfamily,
  morecomment=[s][\color{pblue}]{/**}{*/}
}


\let\pregunta\item

\newcommand*\gqm[3]{
    \begin{itemize}
        \item Objetivos: #1
        \item Preguntas:
        \begin{itemize}
            #2
        \end{itemize}
        \item Métricas:
        \begin{itemize}
            #3
        \end{itemize}
    \end{itemize}
}

\newcommand*\metrica[2]{
    \item Nombre: #1
    \begin{itemize}
        #2
    \end{itemize}
}

\newcommand{\testmetricas}{
    \gqm {
        Definir usabilidad
    } {
        \pregunta pregunta1
        \pregunta pregunta2
    } {
        \metrica {
            Usabildad
        } {
            \item Nombre: Usabildad
            \item Información que brinda: Usabilidad de la aplicación
            \item Para quien es útil: Gerente del proyecto
            \item Entidad: Aplicación
            \item Fórmula: $\Sigma$ items de la heurística cumplidos en su totalidad
            \item Atributos a observar
                \begin{itemize}
                    \item Usabilidad de la aplicación.
                \end{itemize}
            \item Recolección: Heurísticas de usabilidad de Nielsen
            \item Responsable: Desarrollador
            \item Frecuencia: Mensual
            \item Almacenamiento: N/A
            \item Herramientas a utilizar: N/A
        }
    }
}

%Definicion de variables globales

\newcommand{\interesado}{Stakeholder}
\newcommand{\desarrolador}{Desarrollador}
\newcommand{\analista}{Analista}
\newcommand{\pmo}{Gerente de proyecto}

\newenvironment{boxed}
{
    \begin{table}[]
        \begin{tabular}{|p{0.9\textwidth}|}
        \hline\\
}
{ 
        \\\\\hline
        \end{tabular} 
    \end{table}
}

\newcommand{\prettyTable}[2]{
    \begin{table}[h!]
        \resizebox{\textwidth}{!}{
            \begin{tabular}{#1}
                \hline
                #2
            \end{tabular}
        }
    \end{table}
}

\newcommand{\mlcell}[1]{
    \begin{tabular}[c]{@{}l@{}} #1 \end{tabular}
}