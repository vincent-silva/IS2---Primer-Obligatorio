\section{Gestión de los riesgos}

\subsection{Identificación de los riesgos}
\begin{itemize}
    \item \textbf{Requerimientos incorrectos} \\
    \textit{Se manifiesta cuando uno o varios requerimientos no son bien especificados, mal aprobados, o bien, mal detallados por el interesado.}
    \item \textbf{Alteración de los requerimientos} \\
    \textit{En una etapa avanzada del proyecto, se solicita cambiar la especificación de un requerimiento.}
    \item \textbf{Falta de recursos técnicos} \\
    \textit{La no contratación de personal calificado de manera oportuna.}
    \item  \textbf{Falta de recursos materiales} \\
    \textit{Riesgo se manifiesta cuando una tarea depende de un recurso material no presente, o momentáneamente no accesible.}
    \item \textbf{Falla de comunicación en el equipo} \\
    \textit{La incorrecta comunicación entre integrantes del equipo, puede causar trabajo duplicado, conocimiento que no se comparte a oportunamente, entre otros inconvenientes.}
    \item \textbf{Retraso de tareas por previas dependencias incumplidas} \\
    \textit{La dependencia de una tarea por otro evento genera que ese evento pueda afectar el correcto desarrollo de la tarea.}
    \item \textbf{Tiempo de tarea mal estimado} \\
    \textit{La estimación para una tarea es, por ejemplo, extremadamente corta, y genera que un recurso ocupe mucho mas tiempo del pensado en ella.}
    \item \textbf{Priorización errónea de tareas} \\
    \textit{Riesgo que se manifiesta cuando la mala priorizacion errónea altera el orden correcto de las entregas.}
    \item \textbf{Baja de productividad de integrante} \\
    \textit{Un integrante del equipo tiene una productividad muy por debajo de lo espera y utilizado para estimar los tiempos de desarrollo.}
    \item \textbf{Alteración de fecha límite} \\
    \textit{Un cambio de las fechas limite afecta todo el plan de trabajo, pudiendo ser causado por eventos no controlados.}
\end{itemize}
\newpage
\subsection{Categorización de los riesgos}

A continuación se presenta la lista de riesgos, junto con la valoración del impacto y la probabilidad, la cuál fué analizada utilizando el juicio de los miembros del equipo, en función a experiencia análoga con proyectos similares, para cada elemento se diseñó un plan de respuesta para evitar que dicho elemento se convierta en un problema y además se generaron planes de contingencia para amortiguar el efecto nocivo de el riesgo volviendose una realidad. Mas adelante se descibirá el método que se utilizó para definir cuales son los riesgos que necesitan un mayor cuidado.

\prettyTable{|l|l|l|l|l|l|l|}{
    \textbf{Id} & \textbf{Riesgo} & \textbf{\mlcell{Impacto\\(I)}} & \textbf{\mlcell{Probabilidad\\(P)}} & \textbf{\mlcell{Magnitud\\(IxPO)}} & \textbf{Plan de respuesta} & \textbf{Plan de contingencia} \\ \hline
    1 &
    \mlcell{Alteración de los\\requerimientos} 
        & 3 & 0.8 & 2.4 & 
        \mlcell{Validar los requerimientos\\en intervalos regulares} & 
        \mlcell{Revisar, validar y modificar\\el plan de proyecto} \\ \hline
    2 &
    \mlcell{Tiempo de tarea mal estimado}
        & 3 & 0.8 & 2.4 & 
        \mlcell{Evaluar semanalmente el\\ cumplimiento de las tareas\\ y los tiempos requeridos,\\ cotejar con el esfuerzo\\ asignado, actualizar\\ tiempos de tareas del\\ mismo esfuerzo} & 
        \mlcell{Reevaluar prioridades de\\ las tareas} \\ \hline
    3 &
    \mlcell{Baja de productividad\\ del equipo}
        & 3 & 0.6 & 1.8 & 
        \mlcell{Evaluar cada semana el\\ cumplimiento de tareas\\ junto con el tiempo\\ sobrante/faltante} & 
        \mlcell{Cambiar el tiempo\\ estimado para dichas\\ tareas o rechazar tareas\\ futuras de prioridad baja} \\ \hline
    4 &
    \mlcell{Requerimientos \\incorrectos} 
        & 3 & 0.4 & 1.2 &                                       
        \mlcell{Validar los requerimientos\\en intervalos regulares} & 
        \mlcell{Revisar, validar y modificar\\el plan de proyecto} \\ \hline
    5 &
    \mlcell{Dependencias\\ incumplidas de\\ tareas} 
        & 3 & 0.4 & 1.2 & 
        \mlcell{Priorizar las tareas del\\ camino crítico} & 
        \mlcell{Priorizar las tareas del\\ camino crítico} \\ \hline
    6 &
    \mlcell{Falta de recursos\\técnicos} 
        & 4 & 0.2 & 0.8 & 
        \mlcell{Evaluar las tareas a\\realizar antes del inicio y\\generar conocimiento de\\ser necesario} & 
        \mlcell{} \\ \hline
    7 &
    \mlcell{Baja de integrante\\ del equipo} 
        & 4 & 0.2 & 0.8 & 
        \mlcell{Comprometerse a realizar\\ avisos con tiempo sobre\\ situaciones de fuerza\\ mayor} & 
        \mlcell{} \\ \hline
    8 &
    \mlcell{Alteración de los \\criterios de\\ aceptación}
        & 4 & 0.2 & 0.8 & 
        \mlcell{Mantener los criterios\\ de aceptación bien\\ definidos} & 
        \mlcell{Reevaluar las tareas\\ según los criterios\\ de aceptación} \\ \hline
    9 &
    \mlcell{Priorización errónea\\ de tareas} 
        & 1 & 0.6 & 0.6 & 
        \mlcell{Definir correctamente el\\ plan de mantenimiento,\\ asegurarse que la prioridad\\ es la correcta} &  \\ \hline
    10 &
    \mlcell{Falla de\\comunicacion en el\\equipo} 
        & 1 & 0.4 & 0.4 & 
        \mlcell{Evaluar la eficiencia de las\\ vías de comunicación\\ utilizadas, dar cuenta de lo\\ que se realizó y los\\ obstáculos encontrados en\\ cada reunión} & 
        \mlcell{} \\ \hline
    11 &
    \mlcell{Falta de recursos\\materiales} 
        & 1 & 0.2 & 0.2 & 
        \mlcell{Checkear los recursos\\materiales regularmente} &
        \mlcell{} \\ \hline
}

\textbf{Notas}: 
\begin{enumerate}
    \item Los riesgos aparecen ordenados por magnitud.
    \item La columna plan de respuesta corresponde a las tareas a realizar para evitar o amortizar el resigo.
    \item La columna plan de contingencia corresponde a las tareas a realizar en el caso de que el riesgo se genere.
\end{enumerate}

\textbf{Impacto}: 
0 - Ninguno.
1 - Marginal.
2 - Poco importante.
3 - Importante (puede retrasar el proyecto).
4 - Crítica (pude detener el proyecto).
5 - Catastrófica (fracaso del proyecto).	

\textbf{Probabilidad de ocurrencia}:
0.0 - no probable.
0.2 - poco probable.
0.4 - probable.
0.6 - muy probable.
0.8 - altamente probable.
1.0 - se convierte en problema.
\newpage
\subsection{Matriz de riesgos}

A continuacion se detalla la matriz de riesgos, en esta se puede apreciar visualmente como los riesgos van a incidir en nuestro sistema, los colores de las celdas varian segun la suma de los indices del impacto mas el índice de la probabilidad y según el siguiente criterio:

\begin{itemize}
    \item Azul: Poco incidente, en el rango de 0-1
    \item Verde: Cierto grado de incidencia, en el rango de 1-2
    \item Amarillo: Alto grado de incidencia, en el rango de 3-4
    \item Rojo: Grado de incidencia muy alto, en el rango de 8-10
\end{itemize}

Para todos los riesgos de la lista que posean minimamente una incidencia alta se los tomará en consideración y se buscará que los planes de prevención de los mismos se ejecuten correctamente.

\prettyTable{|l|l|l|l|l|l|l|}{
    Seguro              &  \cellcolor{yellow!25} & \cellcolor{orange!25}  & \cellcolor{orange!25}  & \cellcolor{red!25}   & \cellcolor{red!25} &\cellcolor{red!25}  \\ \hline
    Altamente probable  &  \cellcolor{yellow!25} &  \cellcolor{yellow!25} & \cellcolor{orange!25}  & 1 2 \cellcolor{orange!25} & \cellcolor{red!25}  &\cellcolor{red!25} \\ \hline
    Muy probable        & \cellcolor{green!25}  & 9 \cellcolor{yellow!25} &  \cellcolor{yellow!25} & 3 \cellcolor{orange!25} & \cellcolor{orange!25}  & \cellcolor{red!25}\\ \hline
    Probable            & \cellcolor{green!25}  & 10 \cellcolor{green!25} &  \cellcolor{yellow!25} & 4 5 \cellcolor{yellow!25} & \cellcolor{orange!25}  & \cellcolor{orange!25} \\ \hline
    Poco probable       &  \cellcolor{blue!25} & 11 \cellcolor{green!25} &  \cellcolor{green!25} & \cellcolor{yellow!25}  & 6 7 8 \cellcolor{yellow!25} & \cellcolor{orange!25} \\ \hline
    Improbable          & \cellcolor{blue!25}  &  \cellcolor{blue!25} & \cellcolor{green!25}  & \cellcolor{green!25}  & \cellcolor{yellow!25}  & \cellcolor{yellow!25} \\ \hline
                        & Ninguno & Marginal & Poco importante & Importante & Crítico & Catastrófico \\ \hline
}




\begin{comment}




    
    Explicar cómo se va a realizar la gestión de riesgos (identificación, análisis cualitativo, cuantitativo, etc.). 
Máximo: media página.
    
    
    
    
    
    Alteración de fecha límite & 5 & 0 & 0 &  &  \\
    





- Requerimientos mal definidos. (Validar los requerimientos por el cliente).
- No contar todos los recursos necesarios.
- Cambio de requerimientos en una etapa avanzada del proyecto.
- Mala comunicación en el equipo de trabajo.
- Tiempo elevado de aprendizaje de una tecnología.
- Baja de integrante importante del proyecto.
- Priorizar determinadas funcionalidades sobre otras.
- Mala estimación de tiempo y costos.
- Incumplimiento en los estándares de codificación.
- Retraso por dependencia entre tareas.

\end{comment}