\section{Gestión de los riesgos}
Explicar cómo se va a realizar la gestión de riesgos (identificación, análisis cualitativo, cuantitativo, etc.). 
Máximo: media página.

\prettyTable{|l|l|l|l|l|l|}{
    \textbf{Riesgo} & \textbf{Impacto (I)} & \textbf{Probabilidad (P)} & \textbf{Magnitud IxPO} & \textbf{Plan de respuesta} & \textbf{Plan de contingencia} \\ \hline
}


Notas: 
1) Los riesgos aparecen ordenados por magnitud.
2) La columna plan de respuesta se refiere a acciones para que el riesgo no se convierta en problema.
3) La columna plan de contingencia se refiere a acciones en caso que el riesgo se haya convertido en problema.	

Impacto: 
0 - Ninguno.
1 - Marginal.
2 - Poco importante.
3 - Importante (puede retrasar el proyecto).
4 - Crítica (pude detener el proyecto).
5 - Catastrófica (fracaso del proyecto).	

Probabilidad de ocurrencia:
0.0 - no probable.
0.2 - poco probable.
0.4 - probable.
0.6 - muy probable.
0.8 - altamente probable.
1.0 - se convierte en problema.
