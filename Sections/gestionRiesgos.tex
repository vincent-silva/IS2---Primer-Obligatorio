\section{Gestión de los riesgos}

\subsection{Identificación de los riesgos}
\begin{itemize}
    \item Requerimientos incorrectos
    \item Alteración de los requerimientos
    \item Falta de recursos técnicos
    \item Falta de recursos materiales
    \item Falla de comunicación en el equipo
    \item Retraso de tareas por previas dependencias incumplidas
    \item Tiempo de tarea mal estimado
    \item Priorización errónea de tareas
    \item Baja de integrante importante del proyecto
    \item Baja de productividad de integrante
    \item Alteración de fecha límite
\end{itemize}
\newpage
\subsection{Categorización de los riesgos}

\prettyTable{|l|l|l|l|l|l|l|}{
    \textbf{Id} & \textbf{Riesgo} & \textbf{\mlcell{Impacto\\(I)}} & \textbf{\mlcell{Probabilidad\\(P)}} & \textbf{\mlcell{Magnitud\\(IxPO)}} & \textbf{Plan de respuesta} & \textbf{Plan de contingencia} \\ \hline
    1 &
    \mlcell{Alteración de los\\requerimientos} 
        & 3 & 0.8 & 2.4 & 
        \mlcell{Validar los requerimientos\\en intervalos regulares} & 
        \mlcell{Revisar, validar y modificar\\el plan de proyecto} \\ \hline
    2 &
    \mlcell{Tiempo de tarea mal estimado}
        & 3 & 0.8 & 2.4 & 
        \mlcell{Evaluar semanalmente el\\ cumplimiento de las tareas\\ y los tiempos requeridos,\\ cotejar con el esfuerzo\\ asignado, actualizar\\ tiempos de tareas del\\ mismo esfuerzo} & 
        \mlcell{Reevaluar prioridades de\\ las tareas} \\ \hline
    3 &
    \mlcell{Baja de productividad\\ del equipo}
        & 3 & 0.6 & 1.8 & 
        \mlcell{Evaluar cada semana el\\ cumplimiento de tareas\\ junto con el tiempo\\ sobrante/faltante} & 
        \mlcell{Cambiar el tiempo\\ estimado para dichas\\ tareas o rechazar tareas\\ futuras de prioridad baja} \\ \hline
    4 &
    \mlcell{Requerimientos \\incorrectos} 
        & 3 & 0.4 & 1.2 &                                       
        \mlcell{Validar los requerimientos\\en intervalos regulares} & 
        \mlcell{Revisar, validar y modificar\\el plan de proyecto} \\ \hline
    5 &
    \mlcell{Dependencias\\ incumplidas de\\ tareas} 
        & 3 & 0.4 & 1.2 & 
        \mlcell{Priorizar las tareas del\\ camino crítico} & 
        \mlcell{Priorizar las tareas del\\ camino crítico} \\ \hline
    6 &
    \mlcell{Falta de recursos\\técnicos} 
        & 4 & 0.2 & 0.8 & 
        \mlcell{Evaluar las tareas a\\realizar antes del inicio y\\generar conocimiento de\\ser necesario} & 
        \mlcell{Redefinir el tiempo\\disponible para la tarea\\mientras se consiguen los\\recursos técnicos necesarios} \\ \hline
    7 &
    \mlcell{Baja de integrante\\ del equipo} 
        & 4 & 0.2 & 0.8 & 
        \mlcell{Comprometerse a realizar\\ avisos con tiempo sobre\\ situaciones de fuerza\\ mayor} & 
        \mlcell{Priorizar solamente\\ aquellas tareas críticas\\ para el avance del\\ proyecto} \\ \hline
    8 &
    \mlcell{Alteración de los \\criterios de\\ aceptación}
        & 4 & 0.2 & 0.8 & 
        \mlcell{Mantener los criterios\\ de aceptación bien\\ definidos} & 
        \mlcell{Reevaluar las tareas\\ según los criterios\\ de aceptación} \\ \hline
    9 &
    \mlcell{Priorización errónea\\ de tareas} 
        & 1 & 0.6 & 0.6 & 
        \mlcell{Definir correctamente el\\ plan de mantenimiento,\\ asegurarse que la prioridad\\ es la correcta} &  \\ \hline
    10 &
    \mlcell{Falla de\\comunicacion en el\\equipo} 
        & 1 & 0.4 & 0.4 & 
        \mlcell{Evaluar la eficiencia de las\\ vías de comunicación\\ utilizadas, dar cuenta de lo\\ que se realizó y los\\ obstáculos encontrados en\\ cada reunión} & 
        \mlcell{Generar una reunión de\\ actualización para que\\ todos los miembros estén\\ al día} \\ \hline
    11 &
    \mlcell{Falta de recursos\\materiales} 
        & 1 & 0.2 & 0.2 & 
        \mlcell{Checkear los recursos\\materiales regularmente} &
        \mlcell{Buscar reemplazo o\\adquisición de los\\recursos} \\ \hline
}

\textbf{Notas}: 
\begin{enumerate}
    \item Los riesgos aparecen ordenados por magnitud.
    \item La columna plan de respuesta corresponde a las tareas a realizar para evitar o amortizar el resigo.
    \item La columna plan de contingencia corresponde a las tareas a realizar en el caso de que el riesgo se genere.
\end{enumerate}

\textbf{Impacto}: 
0 - Ninguno.
1 - Marginal.
2 - Poco importante.
3 - Importante (puede retrasar el proyecto).
4 - Crítica (pude detener el proyecto).
5 - Catastrófica (fracaso del proyecto).	

\textbf{Probabilidad de ocurrencia}:
0.0 - no probable.
0.2 - poco probable.
0.4 - probable.
0.6 - muy probable.
0.8 - altamente probable.
1.0 - se convierte en problema.
\newpage
\subsection{Matriz de riesgos}
\prettyTable{|l|l|l|l|l|l|l|}{
    Seguro              &  \cellcolor{yellow!25} & \cellcolor{orange!25}  & \cellcolor{orange!25}  & \cellcolor{red!25}   & \cellcolor{red!25} &\cellcolor{red!25}  \\ \hline
    Altamente probable  &  \cellcolor{yellow!25} &  \cellcolor{yellow!25} & \cellcolor{orange!25}  & 1 2 \cellcolor{orange!25} & \cellcolor{red!25}  &\cellcolor{red!25} \\ \hline
    Muy probable        & \cellcolor{green!25}  & 9 \cellcolor{yellow!25} &  \cellcolor{yellow!25} & 3 \cellcolor{orange!25} & \cellcolor{orange!25}  & \cellcolor{red!25}\\ \hline
    Probable            & \cellcolor{green!25}  & 10 \cellcolor{green!25} &  \cellcolor{yellow!25} & 4 5 \cellcolor{yellow!25} & \cellcolor{orange!25}  & \cellcolor{orange!25} \\ \hline
    Poco probable       &  \cellcolor{blue!25} & 11 \cellcolor{green!25} &  \cellcolor{green!25} & \cellcolor{yellow!25}  & 6 7 8 \cellcolor{yellow!25} & \cellcolor{orange!25} \\ \hline
    Improbable          & \cellcolor{blue!25}  &  \cellcolor{blue!25} & \cellcolor{green!25}  & \cellcolor{green!25}  & \cellcolor{yellow!25}  & \cellcolor{yellow!25} \\ \hline
                        & Ninguno & Marginal & Poco importante & Importante & Crítico & Catastrófico \\ \hline
}

\begin{comment}




    
    Explicar cómo se va a realizar la gestión de riesgos (identificación, análisis cualitativo, cuantitativo, etc.). 
Máximo: media página.
    
    
    
    
    
    Alteración de fecha límite & 5 & 0 & 0 &  &  \\
    





- Requerimientos mal definidos. (Validar los requerimientos por el cliente).
- No contar todos los recursos necesarios.
- Cambio de requerimientos en una etapa avanzada del proyecto.
- Mala comunicación en el equipo de trabajo.
- Tiempo elevado de aprendizaje de una tecnología.
- Baja de integrante importante del proyecto.
- Priorizar determinadas funcionalidades sobre otras.
- Mala estimación de tiempo y costos.
- Incumplimiento en los estándares de codificación.
- Retraso por dependencia entre tareas.

\end{comment}