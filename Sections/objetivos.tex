\section{Objetivos}



Para esta aplicación se pone como objetivo que para el dia de la fecha de entrega y utilizando las 22h por semana disponibles en el equipo de desarrollo en su totalidad se cumplan con las siguientes metas:
\begin{itemize}
    \item Se resuelvan el 100\% de los cambios de funcionalidad propuestos en la rúbrica del proyecto
    \item Se detallen el total de los casos de uso requeridos
    \item Se respeten correctamente todas las heurísticas de Nielsen
    \item Se reduzca la cantidad de errores de estilo por debajo de los 50
    \item Se mantenga la cobertura de pruebas unitarias como minimo en 90\%
\end{itemize}
  
\begin{comment}


El objetivo de este proyecto es el de para principios de diciembre, realizar un incremento y mejora de las funcionalidades del software creado para la organización sin fines de lucro \textit{animales sin hogar}, resolviendo todas las funcionalidades nuevas propuestas por el cliente y aumentando en por lo menos un 40\% la calificación sobre los aspectos de calidad que se cumplen pobremente en la aplicación en este momento. Utilizando las 22h por semana disponibles del equipo de desarrollo y realizando mediciones regulares sobre los aspectos de calidad mencionados.

S:
    quien: equipo de desarrollo
    que: aumento de las funcionalidades de la aplicación
    donde: n/a

M:
    - resolución de todas las mejoras propuestas por el cliente
    - mejora en un 40\% de la calificación sobre los aspectos de calidad que se incumplen en la aplicación

A:
    - Utilizando el tiempo 22h por semana disponible del equipo de desarrollo
    - Realizando mediciones regulares de los aspectos de calidad
    
R:
    - 
    
T:
    - Para principios de diciembre

The S.M.A.R.T. acronym stands for:


    S=Smart
    M=Measurable
    A=Achievable
    R=Realistic
    T=Time-bound

And here’s what each part means:


Specific

Make sure your objective is clearly defined. Narrow your scope of the objective so that is has a very tangible and specific outcome. This helps you focus your intent. When writing this part of the objective think of the Who, What, Where, When and Why of it all.


Measurable

Make sure you can actually quantify the objective. If it’s not measurable, you won’t know when the project objective has been met. You want to make sure the objective is trackable to keep you and the team accountable.


Achievable

Make sure you can accomplish the objective. Identify the clear steps that need to happen to make sure the objective is completed. When writing this portion of the objective as yourself how you will accomplish it? What steps need to be taken in order to accomplish the specific objective you’ve defined?


Realistic

This one is really important. Don’t set objectives that can’t be achieved within the constraints of the project. Make sure your objective is practical.  Do you have the budget to do this? Is there enough time? Does your team have the right knowledge or do you have time to invest in learning?


Time-Bound

When will this be done by? Having a clear end date defined helps everybody involved. It lets you know when you need to focus on that objective. It also helps you set a relationship between multiple objectives on a project as well. If you can’t do objective C until A is done and A is getting done in Q1, then you should have C completed in Q2.


\end{comment}