\section{Cronograma de trabajo}

\subsection{Ciclo de vida}

El ciclo de vida mas pertinente para esta etapa del proyecto es una combinación entre el modelo en cascada y un ciclo iterativo incremental, esto se debe a que si bien las tareas se van a organizar de manera que el proyecto avance generando prototipos usables y estables, es en la fecha final que todos los cambios se van a presentar al cliente, en este sentido podría utilizarse únicamente un modelo en cascada, pero, la ventaja de organizar internamente la estructura del trabajo de manera incremental iterativa, viene a darse en el caso de los errores de estimación o los posibles cambios de requerimientos a mitad del proceso de mantenimiento del programa, requieran realizar modificaciones a los requerimientos y por lo tanto al programa, esto genera un mayor control de los riesgos más importantes al proyecto, por esto, se organizarán las tareas de manera que las primeras iteraciones se cumpla con los requisitos críticos del sistema y en posteriores iteraciones se vayan agregando los requerimientos menos importantes.