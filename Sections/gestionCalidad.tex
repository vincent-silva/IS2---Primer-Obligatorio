
\section{Alcance}
En esta sección se describe el plan mediante el cual nos aseguraremos que el proyecto de aseguramiento de calidad y mantenimiento de la aplicación de administración de paseos para la empresa \textit{animales sin hogar}, produzca un producto de calidad, se confeccionarán las tareas, estándares herramientas y métricas para hacer seguimiento al avance del proyecto. Se van a cubrir las áreas de especificación de requerimientos provisto en el proyecto, el plan de proyecto, el código y las pruebas sobre el producto. Desde el inicio del proyecto hasta la entrega del mismo.

\section{Documentos de referencia}
\begin{itemize}
    \item Plan de proyecto descripto anteriormente
    \item Especificación de requerimientos de la aplicación provista
\end{itemize}
\section{Administración}
\subsubsection{Organización}
La asignación de los recursos se provee en el capitulo de organización de RRHH del plan de proyecto.
\subsubsection{Tareas}
\begin{itemize}
    \item Revisiones internas informales
    \item Revisiones técnicas formales
    \item Traceabilidad de los requerimientos
    \item Verificación del sistema, incluyendo testing
    \item Recolección y análisis de las métricas
\end{itemize}

\section{Documentación}
Toda la documentación necesaria está descrita en el plan de proyecto y la especificación de requerimientos

\section{Estándares}
Se detallan los estándares utilizados en el proceso de aseguramiento de calidad del software
\begin{enumerate}
    \item Los documetos entregables producto del avance del proyecto deben seguir el estandar propuesto por el documento 303-FI
    \item El plan de proyecto debe seguir con el estandard provisto por el cuerpo docente
    \item El código debe seguir el estnadard de java
\end{enumerate}

\section{Métricas}
Se van a recolectar periódicamente cierto número de métricas para controlar el avance del proyecto con respecto a los objetivos planteados en el plan de proyecto. Dichas métricas serán recolectadas y analizadas por el encargado de QA las cuales son:

\begin{itemize}
    \item Usabilidad de la aplicación
    \begin{itemize}
        \item Número de items de la eurística cumplidos completamente
    \end{itemize}
    \item Calidad del código
    \begin{itemize}
        \item Cantidad de errores de estilo
        \item Cantidad de métodos por clase
        \item Promedio de cantidad de respuestas por clase
        \item Promedio de cantidad de colaboraciones por clase
        \item Máximo largo de herencia en la aplicación
    \end{itemize}
    \item Completitud de los requerimientos
    \begin{itemize}
        \item Cantidad de requerimientos detallados sobre requerimientos implementados
    \end{itemize}
    \item Correctitud de los casos de uso
    \begin{itemize}
        \item Numero de casos de usos detallados sobre total de caos de uso
    \end{itemize}
    \item Pruebas
    \begin{itemize}
        \item Porcentaje de covertura de las pruebas unitarias
    \end{itemize}
\end{itemize}

\subsection{Usabilidad de la aplicacón}

\gqm {
    % Propósito
    Comprender 
    % Atributo
    la facilidad de uso de de 
    % Objeto
    la aplicación 
    % Punto de vista
    desde el punto de vista del usuario
} {
    \pregunta Qué tan sencillo es el uso de la aplicación?
} {
    \metrica {
        Usabildad
    } {
        \item Nombre: Usabildad
        \item Información que brinda: Usabilidad de la aplicación
        \item Para quien es útil: Gerente del proyecto
        \item Entidad: Aplicación
        \item Fórmula: $\Sigma$ items de la heurística cumplidos en su totalidad
        \item Atributos a observar
            \begin{itemize}
                \item Usabilidad de la aplicación.
            \end{itemize}
        \item Recolección: Heurísticas de usabilidad de Nielsen
        \item Responsable: Desarrollador
        \item Frecuencia: Mensual
        \item Almacenamiento: N/A
        \item Herramientas a utilizar: N/A
    }
}

\subsection{Calidad de código}

\gqm {
    % Propósito
    Mejorar
    % Atributo
    la calidad
    % Objeto
    del código de la aplicación
    % Punto de vista
    desde un punto de vista del código limpio y las buenas prácticas de programación
} {
    \pregunta ¿Que tan bien se puede extender el proyecto?
    \pregunta ¿Que tan fácil es testear?
    \pregunta ¿Que tan bien se puede mantener la aplicación en el tiempo?
    \pregunta ¿Que tanto se puede reutilizar el código existente?
    \pregunta ¿Que tan complejo de entender es el código?
} {
    \metrica {
        \item Nombre: MPC (Cantidad de métodos por clase)
    } {
        \item Información que brinda: Tamaño promedio de las clases
        \item Para quien es útil: Desarrollador
        \item Entidad: Código completo del dominio y de la interfaz
        \item Fórmula: MPC = $\Sigma\ C_i$ (donde $C_i$ es la complejidad ciclomática de una clase)
        \item Atributos a observar
            \begin{itemize}
                \item Extensibilidad
                \item Testeabilidad
            \end{itemize}
        \item Responsable: Desarrollador
        \item Frecuencia: Semanal
        \item Almacenamiento: MS Excel
        \item Herramientas a utilizar:
        \begin{itemize}
            \item IntelliJ Idea
            \item CodeMR
        \end{itemize}
    }
    \metrica {
        \item Nombre: RPC (Respuestas para una clase)
    } {
        \item Información que brinda: 
        \item Para quien es útil: Desarrollador
        \item Entidad: Clases del dominio y la interfaz
        \item Fórmula: RPC = Numero de métodos en el conjunto respuesta
        \item Atributos a observar
            \begin{itemize}
                \item Modificabilidad
                \item Testeabilidad
            \end{itemize}
        \item Responsable: Desarrollador
        \item Frecuencia: Semanal
        \item Almacenamiento: MS Excel
        \item Herramientas a utilizar:
        \begin{itemize}
            \item IntelliJ Idea
            \item CodeMR
        \end{itemize}
    }
    \metrica {
        \item Nombre: ACO (Acoplamiento entre clases de objetos)
    } {
        \item Información que brinda: Cantidad de colaboraciones para una clase determinada
        \item Para quien es útil: Desarrollador
        \item Entidad: Cláses del dominio y la interfaz
        \item Fórmula: ACO =  Numero de colaboraciones
        \item Atributos a observar
            \begin{itemize}
                \item Modificabilidad
                \item Reusabilidad
            \end{itemize}
        \item Responsable: Desarrollador
        \item Frecuencia: Semanal
        \item Almacenamiento: MS Excel
        \item Herramientas a utilizar:
        \begin{itemize}
            \item IntelliJ Idea
            \item CodeMR
        \end{itemize}
    }
    \metrica {
        \item Nombre: PAH (Profundidad del arbol de herencia)
    } {
        \item Información que brinda: Que tan abajo del arbol de herencia se encuentra una apliación
        \item Para quien es útil: Desarrollador
        \item Entidad: Cláses del dominio y la interfaz
        \item Fórmula: PAH = Máxima cantidad de clases por encima de una clase en particular
        \item Atributos a observar
            \begin{itemize}
                \item Complejitud
                \item Mantenibilidad
                \item Extensibilidad
            \end{itemize}
        \item Responsable: Desarrollador
        \item Frecuencia: Semanal
        \item Almacenamiento: MS Excel
        \item Herramientas a utilizar:
        \begin{itemize}
            \item IntelliJ Idea
            \item CodeMR
        \end{itemize}
    }
}

    
    
\subsection{Completitud de los requerimientos}

\gqm {
    % Propósito
    Verificar 
    % Atributo
    la completitud de los requerimientos 
    % Objeto
    del documento 
    % Punto de vista
    desde el punto de vista del \pmo
} {
    \pregunta Existe una referencia de requerimiento en la documentación?
    \pregunta Todos las funciones de la aplicación tienen un requerimiento?
} {
    \metrica {
        Confiabilidad
    } {
        \item Nombre: Confiabilidad
        \item Información que brinda: Confiabilidad de la aplicación
        \item Para quien es útil: \pmo
        \item Entidad: Documentación
        \item Fórmula: Cantidad de requerimientos detallados / requerimientos implementados.
        \item Atributos a observar
            \begin{itemize}
                \item Completitud de cada requerimiento.
            \end{itemize}
        \item Responsable: \analista
        \item Frecuencia: Mensual
        \item Almacenamiento: Informe entregado
        \item Herramientas a utilizar: N/A
    }
}   

\subsection{Casos de uso}

\gqm {
    % Propósito
    Verificar 
    % Atributo
    la cobertura de casos de uso
    % Objeto
    del documento 
    % Punto de vista
    desde el punto de vista del \interesado
} {
    \pregunta Cada una de las funcionalidades de la aplicación tiene un caso de uso?
    \pregunta Los casos de uso reflejan exactamente el flujo de trabajo de cada funcionalidad?
} {
    \metrica {
        Susceptibilidad de someterse a pruebas
    } {
        \item Nombre: Casos de uso
        \item Información que brinda: Susceptibilidad de someterse a pruebas de la aplicación.
        \item Para quien es útil: \pmo
        \item Entidad: Documentación
        \item Fórmula: Cantidad de casos de uso detallados / casos de uso implementados
        \item Atributos a observar
            \begin{itemize}
                \item Completitud
                \item Mantenibilidad
            \end{itemize}
        \item Responsable: \analista
        \item Frecuencia: Mensual
        \item Almacenamiento: Informe entregado
        \item Herramientas a utilizar: N/A
    }
}   

\begin{comment}
\todo{ELIMINAR LUEGO DE TERMINAR}
\begin{itemize}
    \item Plantilla de generación de objetivo
    \item Propósito: Comprender
    \\ Comprender, Controlar, Predecir, Mejorar
    \item Atributo: el tipo y número de defectos
    \item Objeto (Entidad): que se introducen en todas las etapas del desarrollo
    \\Proceso, producto, proyecto
    \item Punto de vista: desde un nivel gerencial
    \\ Stakeholder: Usuario, Cliente, Gerente, Desarrollo, Tester …
    \item Entorno: para los proyectos del cliente X
\end{itemize}

Template:


\begin{itemize}
    \item Meta
\end{itemize}

Preguntas:
\begin{itemize}
    \item Pregunta 1.
    \item Pregunta 2.
\end{itemize}

Métrica:
\begin{itemize}
    \item Métrica 1.
\end{itemize}
\todo{ELIMINAR LUEGO DE TERMINAR}
\end{comment}

\section{Herramientas, técnicas y metodologías}

Las herramientas que utilza el personal encargado de SQA son:
\begin{itemize}
    \item Intellij Idea para realizar los análisis de cobertura de pruebas unitarias
    \item PMD para la extracción de problemas con el estilo de codificación
    \item CodeMR para realizar análisis estático del código
\end{itemize}

Como metodología utilizan las heurísticas de Nielsen para podér analizar los defectos de usabilidad de la aplicación.

\begin{comment}
Describir el plan de calidad.
Máximo: 1 página.


1.
propósito y alcance del plan.
2.
descripción de los productos del trabajo (modelos, documentos,
código fuente, etc.) que se ubiquen dentro del ámbito del QA.
3.
estándares y prácticas aplicables que se utilicen durante el proceso
del software.
4.
acciones y tareas de QA (incluidas revisiones y auditorías) y su
ubicación en el proceso del software.
5.
herramientas y métodos que den apoyo a las acciones y tareas de
ACS (aseguramiento de la calidad del software).
6.
procedimientos para la administración de la configuración del
software.
7.
métodos para mantener todos los registros relacionados con QA.
8.
roles y responsabilidades relacionados con la función de calidad.

IEEE Std 730-1998
a) Purpose;
b) Reference documents;
c) Management;
d) Documentation;
e) Standards, practices, conventions, and metrics;
f) Reviews and audits;
g) Test;
h) Problem reporting and corrective action;
i) Tools, techniques, and methodologies;
j) Code control;
k) Media control;
l) Supplier control;
m) Records collection, maintenance, and retention;
n) Training;
o) Risk management.
\end{comment}