
\section{Alcance}
En esta sección se describe el plan mediante el cual nos aseguraremos que el proyecto de aseguramiento de calidad y mantenimiento de la aplicación de administración de paseos para la empresa \textit{animales sin hogar}, produzca un producto de calidad, se confeccionarán las tareas, estándares herramientas y métricas para hacer seguimiento al avance del proyecto. Se van a cubrir las áreas de especificación de requerimientos provisto en el proyecto, el plan de proyecto, el código y las pruebas sobre el producto. Desde el inicio del proyecto hasta la entrega del mismo.

\section{Documentos de referencia}
\begin{itemize}
    \item Plan de proyecto descripto anteriormente
    \item Especificación de requerimientos de la aplicación provista
\end{itemize}
\section{Administración}
\subsubsection{Organización}
La asignación de los recursos se provee en el capitulo de organización de RRHH del plan de proyecto.
\subsubsection{Tareas}
\begin{itemize}
    \item Revisiones internas informales
    \item Revisiones técnicas formales
    \item Traceabilidad de los requerimientos
    \item Verificación del sistema, incluyendo testing
    \item Recolección y análisis de las métricas
\end{itemize}

\section{Documentación}
Toda la documentación necesaria está descrita en el plan de proyecto y la especificación de requerimientos

\section{Estándares}
Se detallan los estándares utilizados en el proceso de aseguramiento de calidad del software
\begin{enumerate}
    \item Los documetos entregables producto del avance del proyecto deben seguir el estandar propuesto por el documento 303-FI
    \item El plan de proyecto debe seguir con el estandard provisto por el cuerpo docente
    \item El código debe seguir el estnadard de java
\end{enumerate}

\section{Métricas}
Se van a recolectar periódicamente cierto número de métricas para controlar el avance del proyecto con respecto a los objetivos planteados en el plan de proyecto. Dichas métricas serán recolectadas y analizadas por el encargado de QA las cuales son:

\begin{itemize}
    \item Usabilidad de la aplicación
    \begin{itemize}
        \item Número de items de la eurística cumplidos completamente
    \end{itemize}
    \item Calidad del código
    \begin{itemize}
        \item Cantidad de errores de estilo
        \item Cantidad de métodos por clase
        \item Promedio de cantidad de respuestas por clase
        \item Promedio de cantidad de colaboraciones por clase
        \item Máximo largo de herencia en la aplicación
    \end{itemize}
    \item Completitud de los requerimientos
    \begin{itemize}
        \item Cantidad de requerimientos detallados sobre requerimientos implementados
    \end{itemize}
    \item Correctitud de los casos de uso
    \begin{itemize}
        \item Numero de casos de usos detallados sobre total de caos de uso
    \end{itemize}
    \item Pruebas
    \begin{itemize}
        \item Porcentaje de covertura de las pruebas unitarias
    \end{itemize}
\end{itemize}

\section{Herramientas, técnicas y metodologías}

Las herramientas que utilza el personal encargado de SQA son:
\begin{itemize}
    \item Intellij Idea para realizar los análisis de cobertura de pruebas unitarias
    \item PMD para la extracción de problemas con el estilo de codificación
    \item CodeMR para realizar análisis estático del código
\end{itemize}

Como metodología utilizan las heurísticas de Nielsen para podér analizar los defectos de usabilidad de la aplicación.

\begin{comment}
Describir el plan de calidad.
Máximo: 1 página.


1.
propósito y alcance del plan.
2.
descripción de los productos del trabajo (modelos, documentos,
código fuente, etc.) que se ubiquen dentro del ámbito del QA.
3.
estándares y prácticas aplicables que se utilicen durante el proceso
del software.
4.
acciones y tareas de QA (incluidas revisiones y auditorías) y su
ubicación en el proceso del software.
5.
herramientas y métodos que den apoyo a las acciones y tareas de
ACS (aseguramiento de la calidad del software).
6.
procedimientos para la administración de la configuración del
software.
7.
métodos para mantener todos los registros relacionados con QA.
8.
roles y responsabilidades relacionados con la función de calidad.

IEEE Std 730-1998
a) Purpose;
b) Reference documents;
c) Management;
d) Documentation;
e) Standards, practices, conventions, and metrics;
f) Reviews and audits;
g) Test;
h) Problem reporting and corrective action;
i) Tools, techniques, and methodologies;
j) Code control;
k) Media control;
l) Supplier control;
m) Records collection, maintenance, and retention;
n) Training;
o) Risk management.
\end{comment}