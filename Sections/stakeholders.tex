Se identifican los interesados internos y
externos al equipo. Se describe el tipo de
interés, particularmente si es positivo o
negativo. Se realiza un análisis de los objetivos
e influencia de los interesados en el proyecto.


nombre: nombre del interesado
posicion: la posicion que tiene el interesado en la empresa
rol: el rol que tiene el interesado en el equipo de desarrollo
contact information: informacion de contacto del stakeholder, email, telefono, direccion de la cassa
requerimientos: requerimientos de alto nivel para el proyecto y/o el producto
expectativas: principales expectativas para el proyecto y/o el producto
influencia: el grado de influencia que el interesado tiene sobre el proyecto, este puede ser descripto o mediante la utilizacion de una palabra clave
clasificacion: se podria clasificar como amigo, enemigo o neutral. u otra clasificacion como alto medio o bajo impacto


Identification information. Name, organizational position, location and
contact details, and role on the project.
Assessment information. Major requirements, expectations, potential for
influencing project outcomes, and the phase of the project life cycle where the
stakeholder has the most influence or impact.
Stakeholder classification. Internal/external, impact/influence/power/interest,
upward/downward/outward/sideward, or any other classification model chosen
by the project manager.

31.2.1 The Stakeholders
The software process (and every software project) is populated by stakeholders
who can be categorized into one of fi ve constituencies:
1. Senior managers who defi ne the business issues that often have a signifi -
cant infl uence on the project.
2. Project (technical) managers who must plan, motivate, organize, and control
the practitioners who do software work.
3. Practitioners who deliver the technical skills that are necessary to engineer
a product or application.
4. Customers who specify the requirements for the software to be engineered
and other stakeholders who have a peripheral interest in the
outcome.
5. End users

\newpage

\begin{table}[h!]
\begin{tabular}{|l|l|l|l|l|l|l|l|}
\hline
    \textbf{Nombre} & \textbf{Empresa}  & \textbf{Rol}  & \textbf{Información de contacto}  & \textbf{Requerimientos}   & \textbf{Expectativas} & \textbf{Influencia}   & \textbf{Clasificación} \\ \hline
    Vincent Silva   & N/A & Desarrollador
\end{tabular}
\end{table}