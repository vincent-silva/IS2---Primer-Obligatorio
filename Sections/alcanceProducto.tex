\section{Alcance del producto}

Esta incluido en este proyecto los requerimientos funcionales ya existentes que fueron nombrados en el reporte de funcionalidades sin ninguno de las fallas conocidas, añadiendo los requerimientos funcionales y no funcionales solicitados y especificados en este documento.
La totalidad de los requerimientos funcionales que se hacen referencia integran la siguiente lista:

\subsubsection{Requerimientos funcionales:}
\begin{itemize}
    \item \textbf{RF1 - Gestionar la información básica.} \\
    \textit{Permite la gestión apropiada de la información de la mascota, como es:
    su nombre, peso, altura, tipo de animales, posibles comentarios y la imagen.}
    \item \textbf{RF2 - Registrar en calendarios los paseos.} \\
    \textit{Calendario que permite registrar el día y hora de un paseo para una mascota, ingresando el responsable de dicha tarea.}
    \item \textbf{RF3 - Calendario de alimentación de la mascota.} \\
    \textit{Brinda la posibilidad de registrar en el calendario el responsable de alimentar una mascota para un día y hora en especifico.}
    \item \textbf{RF4 - Registro de actividad realizada.} \\
    \textit{Permite registrar una activad ya realizada, la misma puede ser un paseo, alimentación u otra, indicando un nombre, usuario, perro y recorrido o alimento según su tipo.}
    \item \textbf{RF5 - Agendar servicio veterinaria.} \\
    \textit{Funcionalidad que habilita el registar un servicio de cuidado en una veterinaria, como puede ser corte de pelos, uñas, etc.}
    \item \textbf{RF6 - Recordatorios vía email.} \\
    \textit{Envía recordatorio vía correo electrónico y notificaciones de sistema a los responsables de las actividades agendadas previamente.}
    \item \textbf{RF6 - Recordatorios vía email.} \\
    \textit{Carga una colección de datos al sistema para poder operar o probar el resto de su funcionamiento.}
    \item \textbf{RF8 - Agregar usuario.} \\
    \textit{Permite registrar un usuario ingresando nombre y mail, dichos usuarios serán utilizados para la asignación de tareas.}
    \item \textbf{RF9 - Login.} \\
    \textit{Permite al sistema tener control de acceso sobre los usuarios, autorizando únicamente el ingreso a usuarios registrados.}
    \item \textbf{RF10 - Registro de padrinos} \\
    \textit{Brinda la posibilidad a los usuarios de crear nuevos padrinos, registrando nombre, apellido, teléfono, correo electrónico, ciudad, país y asociar las mascotas que desea ser padrino.}

\end{itemize}

\subsubsection{Requerimientos no funcionales:}
\begin{itemize}
    \item \textbf{RNF1 - Usabilidad} \\
    \textit{El sistema cumple con las las heurísticas de Nielsen de:``Libertad y control por parte del usuario" y ``Prevención de errores"}
    \item \textbf{RNF2 - Sin defectos conocidos.} \\
    \textit{El sistema no cuenta con defectos conocidos, siendo los defectos encontrados en las pruebas solucionados antes de la entrega.}
    \item \textbf{RNF3 - Codificación.} \\
    \textit{El sistema no contiene reportes de violaciones a los estándares de java, buenas practicas de programación y ``codestyle".}
    \item \textbf{RNF4 - Caja blanca.} \\
    \textit{El software tiene un 90\% de cobertura de código con pruebas unitarias.}
    \item \textbf{RNF5 - Documentación} \\
    \textit{Todos las funcionalidades del software tienen un ESRE completo asociado.}
\end{itemize}
